\documentclass[letterpaper]{twentysecondcv} % a4paper for A4

% Command for printing skill overview bubbles
\newcommand\skills{ 
~
	\smartdiagram[bubble diagram]{
        \textbf{Full Stack}\\\textbf{~~Web Developer~~'},
        \textbf{REST api}\\\textbf{MEAN stack},
        \textbf{~~~~~Android~~~~~~}\\\textbf{Docker},
        \textbf{~~Angular 4~~}\\\textbf{WebSockets},
        \textbf{~~MongoDB~~}\\\textbf{Postgres},
        \textbf{Graphic}\\\textbf{~~Designing~~},
        \textbf{Firebase}\\\textbf{~~googleapis~~}
    }
}

% Programming skill bars
\programming{
    {C $\textbullet$ C++  $\textbullet$ Golang / 3}, 
    {Express|Node $\textbullet$ Postgres $\textbullet$ MATLAB $\textbullet$ \LaTeX  / 3.5},
    {Angular 4 $\textbullet$ ES6 $\textbullet$ Typescript $\textbullet$ HTML|CSS /5}}


% Projects text
\projects{
\textbf{AIR 1969} in JEE ADVANCED among top 1.5 lakhs selected students of India.\\
\textbf{223/360} in JEE MAINS '16 secured position in top 0.61\% candidates.\\
\textbf{1st} ,consecutive 3 yrs, in City level Maths competition.\\\\
\textbf{Runner-up} in Design-o-flare competition, Takneek’17 ,designed using DS SOLDWORKS 2016



}

%----------------------------------------------------------------------------------------
%	 PERSONAL INFORMATION
%----------------------------------------------------------------------------------------
% If you don't need one or more of the below, just remove the content leaving the command, e.g. \cvnumberphone{}

\cvname{\huge{\textbf{MOHIT KR. SINGH}}} % Your name
\cvjobtitle{Web Developer} % Job
% title/career

\cvlinkedin{}
\cvgithub{MohitKS5}
\cvnumberphone{+91 7318019021} % Phone number
\cvsite{home.iitk.ac.in/~mohitks} % Personal website
\cvmail{mohitks@iitk.ac.in} % Email address

%----------------------------------------------------------------------------------------

\begin{document}

\makeprofile % Print the sidebar

%----------------------------------------------------------------------------------------
%	 EDUCATION
%----------------------------------------------------------------------------------------
\section{Education}

\begin{twenty} % Environment for a list with descriptions
	\twentyitem
    	{2015 - present\t}
        {}
        {BTech., Chemical Engineering \textnormal{(GPA: 8.4/10.0)}}
        {\href{http://www.iitk.ac.in/}{IIT Kanpur, India}}
        {}
        {}
% 	\twentyitem
%     	{}
% 		{}
%         {}{}
%         {}
%         {}
	%\twentyitem{<dates>}{<title>}{<organization>}{<location>}{<description>}
\end{twenty}

\section{Projects}
\begin{twenty}
\twentyitem
    	{Oct 2017 -}
		{Present}
        {\href{http://www.techkriti.org}{Techkriti '18 Website}}
        {College fest}
        {}
        {\begin{itemize}
        \item Currently involved in building a website for Techkriti , one of the largest Tech-Fest in India.
        \end{itemize}}
        \\
    \twentyitem
   		{Apr 2017 -}
		{Oct 2017}
        {\href{http://www.antaragni.in}{Antaragni'17 Website}}
        {\href{http://www.antaragni.in}{college fest}}
        {}
        {
        {\begin{itemize}
        \item Used full \textbf{MEAN} stack to develop the website for college fest.
        \item \textbf{angular 4} framework for modularity, animations and optimization.
        \item Backend coded in \textbf{typescript} provided \textbf{REST} api with \textbf{scalability}.
        \item \textit{Dynamic website} with flexibility in design and content. Element’s arrangement and display could be changed by a GUI.
        \item Provided a user-friendly UI to update the contents of frontend.
    \end{itemize}}
        }
     \\
     \twentyitem
   		{May 2017 -}
		{Jul 2017}
        {IITKMS Website}
        {FSAE team}
        {}
        {
        \begin{itemize}
        \item Built a Web Application from scratch using Angular 4 framework.
        
       % \textit{My work opened up a new position in the organization, enabling it to earn additional revenue of \$3,500 per month (estimated)}
        \item Worked with \textbf{Firebase} for \textit{authentication, cloud storage} and \textit{real–time databasing}.
        \item Used \textbf{Google sheets} as UI for user-friendly \textit{frontend content management} and its api for data fetching.
    \end{itemize}
    	}
    \\
    \twentyitem
   		{March 2017}
		{}
        {Code.Fun.Do}
        {Microsoft}
        {}
        {
        \begin{itemize}
        \item Successfully completed a web app aimed to help college students choose courses and manage them.
        \item Used \textbf{cross-platform Universal App Platform} (Microsoft Azure) for windows 10.
        \item The app simply displayed a filtered list of courses based on user info during registration.
        \item provided \textbf{cloud space} to store course content and relevant links in an organized way just by a \textit{drag and drop}.
    \end{itemize}
    	}
    	
        
	%\twentyitem{<dates>}{<title>}{<location>}{<description>}
\end{twenty}


%----------------------------------------------------------------------------------------
%	 EXPERIENCE
%----------------------------------------------------------------------------------------

\section{Position Of Responsibility}

\begin{twenty} % Environment for a list with descriptions
\twentyitem
    	{OCT 2017 -}
		{Present}
        {\href{http://www.techkriti.org}{Senior Executive,Web, Techkriti '18}}
        {\href{http://www.bell.ca/}{}}
        {}
        {\begin{itemize}
         \item Ensuring the promulgation of the fest through deployment and regular update of website \& app.
        \item Make sure the optimization and accessibility of the website to mass media.
        \end{itemize}
        }
        \\
    \\   
    \twentyitem
   		{Apr 2017 -}
		{Oct 2017}
        {Senior Executive, Web, Antaragni'17}
        {\href{http://www.antaragni.in}{}}
        {}
        {
        {\begin{itemize}
        \item Designed a website for Antaragni, the largest college fest in northern India. 
        \item Provided a portal and maintained the Database of contacts, registered events and payments of about 8000 participants. 
        \item Designed a portal for campus ambassadors and provided them with an online platform linked to social media for publicity of the fest.
    \end{itemize}}
        }
     \\
     \twentyitem
   		{Mar 2017 -}
		{present}
        {Web Designer, FSAE, IITK}
        {\href{http://www.synechron.com/}{}}
        {}
        {
        \begin{itemize}
        \item Exposing the potential of team and project in a representative manner to lure the sponsors and car enthusiasts.
        \item Responsible regular update and functioning of the FSAE website.
    \end{itemize}
    	}
        
	%\twentyitem{<dates>}{<title>}{<location>}{<description>}
\end{twenty}

\end{document} 
